\chapter{Week 4}

\section{k-Means Clustering}

\subsection{Introduction to Clustering}


Hello and welcome. In this video we'll give you a high level introduction to clustering, its applications, and different types of clustering algorithms. Let's get started! Imagine that you have a customer dataset and you need to apply customer segmentation on this historical data. Customer segmentation is the practice of partitioning a customer base into groups of individuals that have similar characteristics. It is a significant strategy, as it allows the business to target specific groups of customers, so as to more effectively allocate marketing resources. For example, one group might contain customers who are high profit and low risk. That is, more likely to purchase products or subscribe for a service. Knowing this information allows a business to devote more time and attention to retaining these customers. Another group might include customers from nonprofit organizations and so on. A general segmentation process is not usually feasible for large volumes of varied data, therefore you need an analytical approach to deriving segments and groups from large datasets. Customers can be grouped based on several factors, including age, gender, interests, spending habits and so on. The important requirement is to use the available data to understand and identify how customers are similar to each other. Let's learn how to divide a set of customers into categories, based on characteristics they share. One of the most adopted approaches that can be used for customer segmentation is clustering. Clustering can group data only unsupervised, based on the similarity of customers to each other. It will partition your customers into mutually exclusive groups. For example, into three clusters. The customers in each cluster are similar to each other demographically. Now we can create a profile for each group, considering the common characteristics of each cluster. For example, the first group made up of affluent and middle aged customers. The second is made up of young, educated and middle income customers, and the third group includes young and low income customers. Finally, we can assign each individual in our dataset to one of these groups or segments of customers. Now imagine that you cross join this segmented dataset with the dataset of the product or services that customers purchase from your company. This information would really help to understand and predict the differences and individual customers preferences and their buying behaviors across various products. Indeed, having this information would allow your company to develop highly personalized experiences for each segment. Customer segmentation is one of the popular usages of clustering. Cluster analysis also has many other applications in different domains. So let's first define clustering and then we'll look at other applications. Clustering means finding clusters in a dataset, unsupervised. So what is a cluster? A cluster is a group of data points or objects in a dataset that are similar to other objects in the group, and dissimilar to datapoints in other clusters. Now the question is," What is different between clustering and classification?" Let's look at our customer dataset again. Classification algorithms predict categorical classed labels. This means assigning instances to predefined classes such as defaulted or not defaulted. For example, if an analyst wants to analyze customer data in order to know which customers might default on their payments, she uses a labeled dataset as training data and uses classification approaches such as a decision tree, Support Vector Machines or SVM, or logistic regression, to predict the default value for a new or unknown customer. Generally speaking, classification is a supervised learning where each training data instance belongs to a particular class. In clustering however, the data is unlabeled and the process is unsupervised. For example, we can use a clustering algorithm such as k-means to group similar customers as mentioned, and assign them to a cluster, based on whether they share similar attributes, such as; age, education, and so on. While I'll be giving you some examples in different industries, I'd like you to think about more samples of clustering. In the retail industry, clustering is used to find associations among customers based on their demographic characteristics and use that information to identify buying patterns of various customer groups. Also, it can be used in recommendation systems to find a group of similar items or similar users and use it for collaborative filtering, to recommend things like books or movies to customers. In banking, analysts find clusters of normal transactions to find the patterns of fraudulent credit card usage. Also they use clustering to identify clusters of customers. For instance, to find loyal customers versus churned customers. In the insurance industry, clustering is used for fraud detection in claims analysis, or to evaluate the insurance risk of certain customers based on their segments. In publication media, clustering is used to auto categorize news based on his content or to tag news, then cluster it so as to recommend similar news articles to readers. In medicine, it can be used to characterize patient behavior, based on their similar characteristics. So as to identify successful medical therapies for different illnesses or in biology, clustering is used to group genes with similar expression patterns or to cluster genetic markers to identify family ties. If you look around you can find many other applications of clustering, but generally clustering can be used for one of the following purposes: exploratory data analysis, summary generation or reducing the scale, outlier detection- especially to be used for fraud detection or noise removal, finding duplicates and datasets or as a pre-processing step for either prediction, other data mining tasks or as part of a complex system. Let's briefly look at different clustering algorithms and their characteristics. Partition-based clustering is a group of clustering algorithms that produces sphere-like clusters, such as; K-Means, K-Medians or Fuzzy c-Means. These algorithms are relatively efficient and are used for medium and large sized databases. Hierarchical clustering algorithms produce trees of clusters, such as agglomerative and divisive algorithms. This group of algorithms are very intuitive and are generally good for use with small size datasets. Density-based clustering algorithms produce arbitrary shaped clusters. They are especially good when dealing with spatial clusters or when there is noise in your data set. For example, the DB scan algorithm. This concludes our video. Thanks for watching! (Music)

\subsection{Introduction to k-Means}


Hello and welcome. In this video, we'll be covering K-Means Clustering. So let's get started. Imagine that you have a customer dataset and you need to apply customer segmentation on this historical data. Customer segmentation is the practice of partitioning a customer base into groups of individuals that have similar characteristics. One of the algorithms that can be used for customer segmentation is K-Means clustering. K-Means can group data only unsupervised based on the similarity of customers to each other. Let's define this technique more formally. There are various types of clustering algorithms such as partitioning, hierarchical or density-based clustering. K-Means is a type of partitioning clustering, that is, it divides the data into K non-overlapping subsets or clusters without any cluster internal structure or labels. This means, it's an unsupervised algorithm. Objects within a cluster are very similar, and objects across different clusters are very different or dissimilar. As you can see, for using K-Means we have to find similar samples: for example, similar customers. Now, we face a couple of key questions. First, how can we find the similarity of samples in clustering, and then how do we measure how similar two customers are with regard to their demographics? Though the objective of K-Means is to form clusters in such a way that similar samples go into a cluster, and dissimilar samples fall into different clusters, it can be shown that instead of a similarity metric, we can use dissimilarity metrics. In other words, conventionally the distance of samples from each other is used to shape the clusters. So we can say K-Means tries to minimize the intra-cluster distances and maximize the inter-cluster distances. Now, the question is, how can we calculate the dissimilarity or distance of two cases such as two customers? Assume that we have two customers, we will call them Customer one and two. Let's also assume that we have only one feature for each of these two customers and that feature is age. We can easily use a specific type of Minkowski distance to calculate the distance of these two customers. Indeed, it is the Euclidean distance. Distance of x1 from x2 is root of 34 minus 30_2 which is four. What about if we have more than one feature, for example age and income. For example, if we have income and age for each customer, we can still use the same formula but this time in a two-dimensional space. Also, we can use the same distance matrix for multidimensional vectors. Of course, we have to normalize our feature set to get the accurate dissimilarity measure. There are other dissimilarity measures as well that can be used for this purpose, but it is highly dependent on datatype and also the domain that clustering is done for it. For example you may use Euclidean distance, Cosine similarity, Average distance, and so on. Indeed, the similarity measure highly controls how the clusters are formed, so it is recommended to understand the domain knowledge of your dataset and datatype of features and then choose the meaningful distance measurement. Now, let's see how K-Means clustering works. For the sake of simplicity, let's assume that our dataset has only two features: the age and income of customers. This means, it's a two-dimensional space. We can show the distribution of customers using a scatter plot: The Y-axis indicates age and the X-axis shows income of customers. We try to cluster the customer dataset into distinct groups or clusters based on these two dimensions. In the first step, we should determine the number of clusters. The key concept of the K-Means algorithm is that it randomly picks a center point for each cluster. It means we must initialize K which represents number of clusters. Essentially, determining the number of clusters in a dataset or K is a hard problem in K-Means, that we will discuss later. For now, let's put K equals three here for our sample dataset. It is like we have three representative points for our clusters. These three data points are called centroids of clusters and should be of same feature size of our customer feature set. There are two approaches to choose these centroids. One, we can randomly choose three observations out of the dataset and use these observations as the initial means. Or two, we can create three random points as centroids of the clusters which is our choice that is shown in the plot with red color. After the initialization step which was defining the centroid of each cluster, we have to assign each customer to the closest center. For this purpose, we have to calculate the distance of each data point or in our case each customer from the centroid points. As mentioned before, depending on the nature of the data and the purpose for which clustering is being used different measures of distance may be used to place items into clusters. Therefore, you will form a matrix where each row represents the distance of a customer from each centroid. It is called the Distance Matrix. The main objective of K-Means clustering is to minimize the distance of data points from the centroid of this cluster and maximize the distance from other cluster centroids. So, in this step, we have to find the closest centroid to each data point. We can use the distance matrix to find the nearest centroid to datapoints. Finding the closest centroids for each data point, we assign each data point to that cluster. In other words, all the customers will fall to a cluster based on their distance from centroids. We can easily say that it does not result in good clusters because the centroids were chosen randomly from the first. Indeed, the model would have a high error. Here, error is the total distance of each point from its centroid. It can be shown as within-cluster sum of squares error. Intuitively, we try to reduce this error. It means we should shape clusters in such a way that the total distance of all members of a cluster from its centroid be minimized. Now, the question is, how can we turn it into better clusters with less error? Okay, we move centroids. In the next step, each cluster center will be updated to be the mean for datapoints in its cluster. Indeed, each centroid moves according to their cluster members. In other words the centroid of each of the three clusters becomes the new mean. For example, if point A coordination is 7.4 and 3.6, and B point features are 7.8 and 3.8, the new centroid of this cluster with two points would be the average of them, which is 7.6 and 3.7. Now, we have new centroids. As you can guess, once again we will have to calculate the distance of all points from the new centroids. The points are reclustered and the centroids move again. This continues until the centroids no longer move. Please note that whenever a centroid moves, each points distance to the centroid needs to be measured again. Yes, K-Means is an iterative algorithm and we have to repeat steps two to four until the algorithm converges. In each iteration, it will move the centroids, calculate the distances from new centroids and assign data points to the nearest centroid. It results in the clusters with minimum error or the most dense clusters. However, as it is a heuristic algorithm, there is no guarantee that it will converge to the global optimum and the result may depend on the initial clusters. It means, this algorithm is guaranteed to converge to a result, but the result may be a local optimum i.e. not necessarily the best possible outcome. To solve this problem, it is common to run the whole process multiple times with different starting conditions. This means with randomized starting centroids, it may give a better outcome. As the algorithm is usually very fast, it wouldn't be any problem to run it multiple times. Thanks for watching this video.

\subsection{More on k-Means}

Hello and welcome. In this video, we'll look at k-Means accuracy and characteristics. Let's get started. Let's define the algorithm more concretely, before we talk about its accuracy.
Play video starting at ::14 and follow transcript0:14
A k-Means algorithm works by randomly placing k centroids, one for each cluster. The farther apart the clusters are placed, the better. The next step is to calculate the distance of each data point or object from the centroids. Euclidean distance is used to measure the distance from the object to the centroid. Please note, however, that you can also use different types of distance measurements, not just Euclidean distance. Euclidean distance is used because it's the most popular. Then, assign each data point or object to its closest centroid creating a group. Next, once each data point has been classified to a group, recalculate the position of the k centroids. The new centroid position is determined by the mean of all points in the group. Finally, this continues until the centroids no longer move. Now, the questions is, how can we evaluate the goodness of the clusters formed by k-Means? In other words, how do we calculate the accuracy of k-Means clustering? One way is to compare the clusters with the ground truth, if it's available. However, because k-Means is an unsupervised algorithm we usually don't have ground truth in real world problems to be used. But there is still a way to say how bad each cluster is, based on the objective of the k-Means. This value is the average distance between data points within a cluster. Also, average of the distances of data points from their cluster centroids can be used as a metric of error for the clustering algorithm. Essentially, determining the number of clusters in a data set, or k as in the k-Means algorithm, is a frequent problem in data clustering. The correct choice of K is often ambiguous because it's very dependent on the shape and scale of the distribution of points in a dataset. There are some approaches to address this problem, but one of the techniques that is commonly used is to run the clustering across the different values of K and looking at a metric of accuracy for clustering.
Play video starting at :2:30 and follow transcript2:30
This metric can be mean, distance between data points and their cluster's centroid, which indicate how dense our clusters are or, to what extent we minimize the error of clustering. Then, looking at the change of this metric, we can find the best value for K.
Play video starting at :2:49 and follow transcript2:49
But the problem is that with increasing the number of clusters, the distance of centroids to data points will always reduce. This means increasing K will always decrease the error. So, the value of the metric as a function of K is plotted and the elbow point is determined where the rate of decrease sharply shifts. It is the right K for clustering. This method is called the elbow method. So let's recap k-Means clustering: k-Means is a partition-based clustering which is A, relatively efficient on medium and large sized data sets; B, produces sphere-like clusters because the clusters are shaped around the centroids; and C, its drawback is that we should pre-specify the number of clusters, and this is not an easy task. Thanks for watching. (Music)


\section{Hierarchical Clustering}

\subsection{Intro to Hierarchical Clustering}


Hello and welcome. In this video, we'll be covering Hierarchical Clustering. So, let's get started. Let's look at this chart. An international team of scientists led by UCLA biologists used this dendrogram to report genetic data from more than 900 dogs from 85 breeds, and more than 200 wild gray wolves worldwide, including populations from North America, Europe, the Middle East, and East Asia. They used molecular genetic techniques to analyze more than 48,000 genetic markers. This diagram shows hierarchical clustering of these animals based on the similarity in their genetic data. Hierarchical clustering algorithms build a hierarchy of clusters where each node is a cluster consisting of the clusters of its daughter nodes. Strategies for hierarchical clustering generally fall into two types, divisive and agglomerative. Divisive is top down, so you start with all observations in a large cluster and break it down into smaller pieces. Think about divisive as dividing the cluster. Agglomerative is the opposite of divisive. So it is bottom up, where each observation starts in its own cluster and pairs of clusters are merged together as they move up the hierarchy. Agglomeration means to amass or collect things, which is exactly what this does with the cluster. The agglomerative approach is more popular among data scientists and so it is the main subject of this video. Let's look at a sample of agglomerative clustering. This method builds the hierarchy from the individual elements by progressively merging clusters. In our example, let's say we want to cluster six cities in Canada based on their distances from one another. They are Toronto, Ottawa, Vancouver, Montreal, Winnipeg, and Edmonton. We construct a distance matrix at this stage, where the numbers in the row i column j is the distance between the i and j cities. In fact, this table shows the distances between each pair of cities. The algorithm is started by assigning each city to its own cluster. So if we have six cities, we have six clusters each containing just one city. Let's note each city by showing the first two characters of its name. The first step is to determine which cities, let's call them clusters from now on, to merge into a cluster. Usually, we want to take the two closest clusters according to the chosen distance. Looking at the distance matrix, Montreal and Ottawa are the closest clusters so we make a cluster out of them. Please notice that we just use a simple one-dimensional distance feature here, but our object can be multidimensional and distance measurement can either be Euclidean, Pearson, average distance or many others depending on data type and domain knowledge. Anyhow, we have to merge these two closest cities in the distance matrix as well. So, rows and columns are merged as the cluster is constructed. As you can see in the distance matrix, rows and columns related to Montreal and Ottawa cities are merged as the cluster is constructed. Then, the distances from all cities to this new merged cluster get updated. But how? For example, how do we calculate the distance from Winnipeg to the Ottawa/Montreal cluster? Well, there are different approaches but let's assume for example, we just select the distance from the center of the Ottawa/Montreal cluster to Winnipeg. Updating the distance matrix, we now have one less cluster. Next, we look for the closest clusters once again. In this case, Ottawa, Montreal, and Toronto are the closest ones which creates another cluster. In the next step, the closest distances between the Vancouver cluster and the Edmonton cluster. Forming a new cluster, the data in the matrix table gets updated. Essentially, the rows and columns are merged as the clusters are merged and the distance updated. This is a common way to implement this type of clustering and has the benefit of caching distances between clusters. In the same way, agglomerative algorithm proceeds by merging clusters, and we repeat it until all clusters are merged and the tree becomes completed. It means, until all cities are clustered into a single cluster of size six. Hierarchical clustering is typically visualized as a dendrogram as shown on this slide. Each merge is represented by a horizontal line. The y-coordinate of the horizontal line is the similarity of the two clusters that were merged where cities are viewed as singleton clusters. By moving up from the bottom layer to the top node, a dendrogram allows us to reconstruct the history of merges that resulted in the depicted clustering. Essentially, hierarchical clustering does not require a prespecified number of clusters. However, in some applications, we want a partition of disjoint clusters just as in flat clustering. In those cases, the hierarchy needs to be cut at some point. For example here, cutting in a specific level of similarity, we create three clusters of similar cities. This concludes this video. Thanks for watching. (Music)

\subsection{More on Hierarchical Clustering}


Hello and welcome. In this video, we'll be covering more details about hierarchical clustering. Let's get started. Let's look at agglomerative algorithm for hierarchical clustering.
Play video starting at ::13 and follow transcript0:13
Remember that agglomerative clustering is a bottom up approach.
Play video starting at ::18 and follow transcript0:18
Let's say our data set has n data points. First, we want to create n clusters, one for each data point. Then, each point is assigned as a cluster.
Play video starting at ::31 and follow transcript0:31
Next, we want to compute the distance proximity matrix which will be an n by n table. After that, we want to iteratively run the following steps until the specified cluster number is reached, or until there is only one cluster left. First, merge the two nearest clusters. Distances are computed already in the proximity matrix. Second, update the proximity matrix with the new values. We stop after we've reached the specified number of clusters, or there is only one cluster remaining with the result stored in a dendogram.
Play video starting at :1:11 and follow transcript1:11
So in the proximity matrix, we have to measure the distances between clusters and also merge the clusters that are nearest.
Play video starting at :1:21 and follow transcript1:21
So, the key operation is the computation of the proximity between the clusters with one point and also clusters with multiple data points. At this point, there are a number of key questions that need to be answered. For instance, how do we measure the distances between these clusters and how do we define the nearest among clusters? We also can ask, which points do we use? First, let's see how to calculate the distance between two clusters with one point each.
Play video starting at :1:54 and follow transcript1:54
Let's assume that we have a data set of patients and we want to cluster them using hierarchy clustering. So our data points are patients with a featured set of three dimensions. For example, age, body mass index, or BMI and blood pressure. We can use different distance measurements to calculate the proximity matrix. For instance, Euclidean distance.
Play video starting at :2:21 and follow transcript2:21
So, if we have a data set of n patience, we can build an n by n dissimilarity distance matrix. It will give us the distance of clusters with one data point.
Play video starting at :2:34 and follow transcript2:34
However, as mentioned, we merge clusters in agglomerative clustering. Now the question is, how can we calculate the distance between clusters when there are multiple patients in each cluster?
Play video starting at :2:48 and follow transcript2:48
We can use different criteria to find the closest clusters and merge them. In general, it completely depends on the data type, dimensionality of data and most importantly, the domain knowledge of the data set. In fact, different approaches to defining the distance between clusters distinguish the different algorithms. As you might imagine, there are multiple ways we can do this. The first one is called single linkage clustering. Single linkage is defined as the shortest distance between two points in each cluster, such as point a and b.
Play video starting at :3:27 and follow transcript3:27
Next up is complete linkage clustering. This time, we are finding the longest distance between the points in each cluster, such as the distance between point a and b.
Play video starting at :3:39 and follow transcript3:39
The third type of linkage is average linkage clustering or the mean distance. This means we're looking at the average distance of each point from one cluster to every point in another cluster.
Play video starting at :3:53 and follow transcript3:53
The final linkage type to be reviewed is centroid linkage clustering. Centroid is the average of the feature sets of points in a cluster. This linkage takes into account the centroid of each cluster when determining the minimum distance.
Play video starting at :4:11 and follow transcript4:11
There are three main advantages to using hierarchical clustering. First, we do not need to specify the number of clusters required for the algorithm. Second, hierarchical clustering is easy to implement. And third, the dendrogram produced is very useful in understanding the data.
Play video starting at :4:32 and follow transcript4:32
There are some disadvantages as well. First, the algorithm can never undo any previous steps. So for example, the algorithm clusters two points and later on, we see that the connection was not a good one. The program can not undo that step.
Play video starting at :4:51 and follow transcript4:51
Second, the time complexity for the clustering can result in very long computation times in comparison with efficient algorithms such as K-means. Finally, if we have a large data set, it can become difficult to determine the correct number of clusters by the dendrogram.
Play video starting at :5:10 and follow transcript5:10
Now, lets compare hierarchical clustering with K-means. K-means is more efficient for large data sets. In contrast to K-means, hierarchical clustering does not require the number of cluster to be specified. Hierarchical clustering gives more than one partitioning depending on the resolution or as K-means gives only one partitioning of the data. Hierarchical clustering always generates the same clusters, in contrast with K-means, that returns different clusters each time it is run, due to random initialization of centroids.
Play video starting at :5:49 and follow transcript5:49
Thanks for watching.


\section{Density-based Clustering}

\subsection{DBSCAN}

Hello and welcome. In this video, we'll be covering DB scan. A density-based clustering algorithm which is appropriate to use when examining spatial data. So let's get started. Most of the traditional clustering techniques such as K-Means, hierarchical, and Fuzzy clustering can be used to group data in an unsupervised way. However, when applied to tasks with arbitrary shaped clusters or clusters within clusters, traditional techniques might not be able to achieve good results, that is elements in the same cluster might not share enough similarity or the performance may be poor. Additionally, while partitioning based algorithms such as K-Means may be easy to understand and implement in practice, the algorithm has no notion of outliers that is, all points are assigned to a cluster even if they do not belong in any. In the domain of anomaly detection, this causes problems as anomalous points will be assigned to the same cluster as normal data points. The anomalous points pull the cluster centroid towards them making it harder to classify them as anomalous points. In contrast, density-based clustering locates regions of high density that are separated from one another by regions of low density. Density in this context is defined as the number of points within a specified radius. A specific and very popular type of density-based clustering is DBSCAN. DBSCAN is particularly effective for tasks like class identification on a spatial context. The wonderful attributes of the DBSCAN algorithm is that it can find out any arbitrary shaped cluster without getting effected by noise. For example, this map shows the location of weather stations in Canada. DBSCAN can be used here to find the group of stations which show the same weather condition. As you can see, it not only finds different arbitrary shaped clusters it can find the denser part of data-centered samples by ignoring less dense areas or noises. Now, let's look at this clustering algorithm to see how it works. DBSCAN stands for Density-Based Spatial Clustering of Applications with Noise. This technique is one of the most common clustering algorithms which works based on density of object. DBSCAN works on the idea that if a particular point belongs to a cluster it should be near to lots of other points in that cluster. It works based on two parameters: radius and minimum points. R determines a specified radius that if it includes enough points within it, we call it a dense area. M determines the minimum number of data points we want in a neighborhood to define a cluster. Let's define radius as two units. For the sake of simplicity, assume it has radius of two centimeters around a point of interest. Also, let's set the minimum point or M to be six points including the point of interest. To see how DBSCAN works, we have to determine the type of points. Each point in our dataset can be either a core, border, or outlier point. Don't worry, I'll explain what these points are in a moment. But the whole idea behind the DBSCAN algorithm is to visit each point and find its type first, then we group points as clusters based on their types. Let's pick a point randomly. First, we check to see whether it's a core data point. So, what is a core point? A data point is a core point if within our neighborhood of the point there are at least M points. For example, as there are six points in the two centimeter neighbor of the red point, we mark this point as a core point. Okay, what happens if it's not a core point? Let's look at another point. Is this point a core point? No. As you can see, there are only five points in this neighborhood including the yellow point, so what kind of point is this one? In fact, it is a border point. What is a border point? A data point is a border point if A; its neighbourhood contains less than M data points or B; it is reachable from some core point. Here, reachability means it is within our distance from a core point. It means that even though the yellow point is within the two centimeter neighborhood of the red point, it is not by itself a core point because it does not have at least six points in its neighborhood. We continue with the next point. As you can see, it is also a core point and all points around it which are not core points are border points. Next core point and next core point. Let's pick this point. You can see it is not a core point nor is it a border point. So, we'd label it as an outlier. What is an outlier? An outlier is a point that is not a core point and also is not close enough to be reachable from a core point. We continue and visit all the points in the dataset and label them as either core, border, or outlier. The next step is to connect core points that are neighbors and put them in the same cluster. So, a cluster is formed as at least one core point plus all reachable core points plus all their borders. It's simply shapes all the clusters and finds outliers as well. Let's review this one more time to see why DBSCAN is cool. DBSCAN can find arbitrarily shaped clusters. It can even find a cluster completely surrounded by a different cluster. DBSCAN has a notion of noise and is robust to outliers. On top of that, DBSCAN makes it very practical for use in many real-world problems because it does not require one to specify the number of clusters such as K in K-means. This concludes this video. 

