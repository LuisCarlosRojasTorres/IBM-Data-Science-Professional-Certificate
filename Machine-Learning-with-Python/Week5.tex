\chapter{Week 5}

\section{Content-based Recommendation Engines}

	\subsection{Intro to Recommender Systems}
	
	
	Hello, and welcome! In this video, we’ll be going through a quick introduction to recommendation systems. So, let’s get started. Even though peoples’ tastes may vary, they generally follow patterns. By that, I mean that there are similarities in the things that people tend to like … or another way to look at it, is that people tend to like things in the same category or things that share the same characteristics. For example, if you’ve recently purchased a book on Machine Learning in Python and you’ve enjoyed reading it, it’s very likely that you’ll also enjoy reading a book on Data Visualization. People also tend to have similar tastes to those of the people they’re close to in their lives. Recommender systems try to capture these patterns and similar behaviors, to help predict what else you might like. Recommender systems have many applications that I’m sure you’re already familiar with. Indeed, Recommender systems are usually at play on many websites. For example, suggesting books on Amazon and movies on Netflix. In fact, everything on Netflix’s website is driven by customer selection. If a certain movie gets viewed frequently enough, Netflix’s recommender system ensures that that movie gets an increasing number of recommendations. Another example can be found in a daily-use mobile app, where a recommender engine is used to recommend anything from where to eat or what job to apply to. On social media, sites like Facebook or LinkedIn, regularly recommend friendships. Recommender systems are even used to personalize your experience on the web. For example, when you go to a news platform website, a recommender system will make note of the types of stories that you clicked on and make recommendations on which types of stories you might be interested in reading in future. There are many of these types of examples and they are growing in number every day. So, let’s take a closer look at the main benefits of using a recommendation system. One of the main advantages of using recommendation systems is that users get a broader exposure to many different products they might be interested in. This exposure encourages users towards continual usage or purchase of their product. Not only does this provide a better experience for the user but it benefits the service provider, as well, with increased potential revenue and better security for its customers. There are generally 2 main types of recommendation systems: Content-based and collaborative filtering. The main difference between each, can be summed up by the type of statement that a consumer might make. For instance, the main paradigm of a Content-based recommendation system is driven by the statement: “Show me more of the same of what I've liked before." Content-based systems try to figure out what a user's favorite aspects of an item are, and then make recommendations on items that share those aspects. Collaborative filtering is based on a user saying, “Tell me what's popular among my neighbors because I might like it too.” Collaborative filtering techniques find similar groups of users, and provide recommendations based on similar tastes within that group. In short, it assumes that a user might be interested in what similar users are interested in. Also, there are Hybrid recommender systems, which combine various mechanisms. In terms of implementing recommender systems, there are 2 types: Memory-based and Model-based. In memory-based approaches, we use the entire user-item dataset to generate a recommendation system. It uses statistical techniques to approximate users or items. Examples of these techniques include: Pearson Correlation, Cosine Similarity and Euclidean Distance, among others. In model-based approaches, a model of users is developed in an attempt to learn their preferences. Models can be created using Machine Learning techniques like regression, clustering, classification, and so on. This is the end of our video. Thanks for watching! (Music)
	
	\subsection{Content-based Recommender Systems}	
	
	
	Hello, and welcome. In this video, we'll be covering Content-Based Recommender Systems. So let's get started. A Content-based recommendation system tries to recommend items to users based on their profile. The user's profile revolves around that user's preferences and tastes. It is shaped based on user ratings, including the number of times that user has clicked on different items or perhaps even liked those items. The recommendation process is based on the similarity between those items. Similarity or closeness of items is measured based on the similarity in the content of those items. When we say content, we're talking about things like the items category, tag, genre, and so on. For example, if we have four movies, and if the user likes or rates the first two items, and if Item 3 is similar to Item 1 in terms of their genre, the engine will also recommend Item 3 to the user. In essence, this is what content-based recommender system engines do. Now, let's dive into a content-based recommender system to see how it works. Let's assume we have a data set of only six movies. This data set shows movies that our user has watched and also the genre of each of the movies. For example, Batman versus Superman is in the Adventure, Super Hero genre and Guardians of the Galaxy is in the Comedy, Adventure, Super Hero and Science-fiction genres. Let's say the user has watched and rated three movies so far and she has given a rating of two out of 10 to the first movie, 10 out of 10 to the second movie and eight out of 10 to the third. The task of the recommender engine is to recommend one of the three candidate movies to this user, or in other, words we want to predict what the user's possible rating would be of the three candidate movies if she were to watch them. To achieve this, we have to build the user profile. First, we create a vector to show the user's ratings for the movies that she's already watched. We call it Input User Ratings. Then, we encode the movies through the one-hot encoding approach. Genre of movies are used here as a feature set. We use the first three movies to make this matrix, which represents the movie feature set matrix. If we multiply these two matrices we can get the weighted feature set for the movies. Let's take a look at the result. This matrix is also called the Weighted Genre matrix and represents the interests of the user for each genre based on the movies that she's watched. Now, given the Weighted Genre Matrix, we can shape the profile of our active user. Essentially, we can aggregate the weighted genres and then normalize them to find the user profile. It clearly indicates that she likes superhero movies more than other genres. We use this profile to figure out what movie is proper to recommend to this user. Recall that we also had three candidate movies for recommendation that haven't been watched by the user, we encode these movies as well. Now we're in the position where we have to figure out which of them is most suited to be recommended to the user. To do this, we simply multiply the User Profile matrix by the candidate Movie Matrix, which results in the Weighted Movies Matrix. It shows the weight of each genre with respect to the User Profile. Now, if we aggregate these weighted ratings, we get the active user's possible interest level in these three movies. In essence, it's our recommendation lists, which we can sort to rank the movies and recommend them to the user. For example, we can say that the Hitchhiker's Guide to the Galaxy has the highest score in our list, and it's proper to recommend to the user. Now, you can come back and fill the predicted ratings for the user. So, to recap what we've discussed so far, the recommendation in a content-based system is based on user's taste and the content or feature set items. Such a model is very efficient. However, in some cases, it doesn't work. For example, assume that we have a movie in the drama genre, which the user has never watch. So, this genre would not be in her profile. Therefore, shall only get recommendations related to genres that are already in her profile and the recommender engine may never recommend any movie within other genres. This problem can be solved by other types of recommender systems such as collaborative filtering. Thanks for watching.
		
	\subsection{Collaborative Filtering}
	
	
	Hello, and welcome. In this video, we'll be covering a recommender system technique called collaborative filtering. So let's get started. Collaborative filtering is based on the fact that relationships exist between products and people's interests. Many recommendation systems use collaborative filtering to find these relationships and to give an accurate recommendation of a product that the user might like or be interested in. Collaborative filtering has basically two approaches: user-based and item-based. User-based collaborative filtering is based on the user similarity or neighborhood. Item-based collaborative filtering is based on similarity among items. Let's first look at the intuition behind the user-based approach. In user-based collaborative filtering, we have an active user for whom the recommendation is aimed. The collaborative filtering engine first looks for users who are similar. That is users who share the active users rating patterns. Collaborative filtering basis this similarity on things like history, preference, and choices that users make when buying, watching, or enjoying something. For example, movies that similar users have rated highly. Then it uses the ratings from these similar users to predict the possible ratings by the active user for a movie that she had not previously watched. For instance, if two users are similar or are neighbors in terms of their interested movies, we can recommend a movie to the active user that her neighbor has already seen. Now, let's dive into the algorithm to see how all of this works. Assume that we have a simple user item matrix, which shows the ratings of four users for five different movies. Let's also assume that our active user has watched and rated three out of these five movies. Let's find out which of the two movies that our active user hasn't watched should be recommended to her. The first step is to discover how similar the active user is to the other users. How do we do this? Well, this can be done through several different statistical and vectorial techniques such as distance or similarity measurements including Euclidean Distance, Pearson Correlation, Cosine Similarity, and so on. To calculate the level of similarity between two users, we use the three movies that both the users have rated in the past. Regardless of what we use for similarity measurement, let's say for example, the similarity could be 0.7, 0.9, and 0.4 between the active user and other users. These numbers represent similarity weights or proximity of the active user to other users in the dataset. The next step is to create a weighted rating matrix. We just calculated the similarity of users to our active user in the previous slide. Now, we can use it to calculate the possible opinion of the active user about our two target movies. This is achieved by multiplying the similarity weights to the user ratings. It results in a weighted ratings matrix, which represents the user's neighbors opinion about are two candidate movies for recommendation. In fact, it incorporates the behavior of other users and gives more weight to the ratings of those users who are more similar to the active user. Now, we can generate the recommendation matrix by aggregating all of the weighted rates. However, as three users rated the first potential movie and two users rated the second movie, we have to normalize the weighted rating values. We do this by dividing it by the sum of the similarity index for users. The result is the potential rating that our active user will give to these movies based on her similarity to other users. It is obvious that we can use it to rank the movies for providing recommendation to our active user. Now, let's examine what's different between user-based and item-based collaborative filtering. In the user-based approach, the recommendation is based on users of the same neighborhood with whom he or she shares common preferences. For example, as User 1 and User 3 both liked Item 3 and Item 4, we consider them as similar or neighbor users, and recommend Item 1 which is positively rated by User 1 to User 3. In the item-based approach, similar items build neighborhoods on the behavior of users. Please note however, that it is not based on their contents. For example, Item 1 and Item 3 are considered neighbors as they were positively rated by both User 1 and User 2. So, Item 1 can be recommended to User 3 as he has already shown interest in Item 3. Therefore, the recommendations here are based on the items in the neighborhood that a user might prefer. Collaborative filtering is a very effective recommendation system. However, there are some challenges with it as well. One of them is data sparsity. Data sparsity happens when you have a large data set of users who generally rate only a limited number of items. As mentioned, collaborative based recommenders can only predict scoring of an item if there are other users who have rated it. Due to sparsity, we might not have enough ratings in the user item dataset which makes it impossible to provide proper recommendations. Another issue to keep in mind is something called cold start. Cold start refers to the difficulty the recommendation system has when there is a new user, and as such a profile doesn't exist for them yet. Cold start can also happen when we have a new item which has not received a rating. Scalability can become an issue as well. As the number of users or items increases and the amount of data expands, collaborative filtering algorithms will begin to suffer drops in performance, simply due to growth and the similarity computation. There are some solutions for each of these challenges such as using hybrid based recommender systems, but they are out of scope of this course. Thanks for watching. (Music)
	
	