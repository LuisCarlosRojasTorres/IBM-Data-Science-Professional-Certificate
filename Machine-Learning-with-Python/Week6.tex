\chapter{Week 6}

\section{Final Project}

	\subsection{OPTIONAL: Signing-up for a Watson Studio Account}
	
	
	All right, before we start learning about foursquare and their API, let's get some of the logistics out of the way. For this project, we will be using Watson Studio as the platform to build our code for the project and to share it with our peers. So each of you will need to create a Watson Studio account in order to proceed with the course. And in this video I will walk you through the process. Now for those you who don't know, Watson Studio is an awesome platform that provides a suite of tools for data other scientists to collaboratively and easily work with application developers and subject matter experts. And to build, train, and deploy models at scale. For the sake of this project, we won't be doing any extensive machine learning. But by using Watson Studio and learning about its features, you will hopefully appreciate what it has to offer. So let's go to the following link that is www.ibm.com/cloud/watson-studio. This will take you to the landing page of Watson Studio. Feel free to check it out to learn more about its features and what the platform has to offer. You can also watch the video to learn more about it. And once you're ready to create an account, you can simply click on Start your free trial. This will bring you to a page where you have to enter your email address. So I'm going to do that. And then check the box to accept the terms. Once you've done that, click Next. This will redirect you to a page where you have to fill in your personal details. So go ahead and fill in your First and Last Name, your Country or Region, and a password for your account. Then you can check the box if you wish to remain informed about any products or services by IBM. And then scroll down and click Create Account. Awesome, so at this point, you should receive an email to confirm your account and to continue with this kind of process. So go to your account and follow the instructions in the email. I'm going to do the same thing here. [SOUND] By the way, this is just a dummy account that I created for demonstration purposes. There you go. Here is the email from IBM Cloud. So you click on the email and you click Confirm Account. Awesome. You should automatically then get redirected to this page. Now this page has all the details about the IBM privacy policy, so feel free to read it to understand the policy and your rights as a user. Then scroll down all the way to the bottom and click proceed. Here you can specify what account to use and the organization. And the IBM cloud space, in case you have more than one. But for most of you, you can simply go with the default settings, so just click Continue. Right now we'll take some time to create your account. There you go. And that completes the process of signing up on Watson Studio. Now you can click Get Started. Once you get to this page,you're good to go and you're good to start creating new projects. Now in the next video, I'll show you how to create a project and how to start creating notebooks under that project and how to share them with your peers or other collaborators. So I'll see you in the next video
	
	\subsection{OPTIONAL: Sharing Notebooks on Watson Studio}
	
	
	All right, so in this video, I will walk you through the process of creating a project on Watson Studio for discourse, and I'll show you how to create notebooks under that project and how to share them with your peers, since your submissions in this course will be mainly notebooks shared from your Watson Studio account. It's very important that you familiarize yourself with this process. Let's go to Watson Studio, and log into your account. When you click sign in, you should land on this page. This is the landing page of your account. Here you can see a summary and overview of your account for example, any projects that you created or are currently working on, any catalogs that you created so you can organize your assets and your projects into catalogs, or any Watson services that you signed up for and you're currently using. But since you just created your account, then you shouldn't have any projects listed. To create a project, you can simply click on the New project icon, so you click on that. Watson Studio supports different types of projects ranging from very basic projects to very complicated ones involving Deep Learning and Visual Recognition and Data Engineering. The beautiful thing about Watson Studio is that the environment for each project is prepped for you, meaning each project comes with a suite of tools and services pre-installed for your convenience so you don't have to worry and spend time installing them. If you expect to use all of the different concepts shown here, then you can start a complete project. But in our case, we're simply analyzing neighborhood data and we're not doing any extensive machine learning work, so creating a Data Science project should suffice. Let's click on Data Science and click okay. Now here, we get to name our project and briefly describe it for a reference or friendly collaborators who will be joining anytime in the future. Let's name our project Coursera_ Capstone and give it a brief description. I'm going to use Coursera Capstone Project- Analyzing Neighborhood Data.
	Play video starting at :3:11 and follow transcript3:11
	Now for this option, if you'll be sharing your code with any external collaborator, which in this case you will since you will be sharing the code with your peers for peer evaluation and assessment, then you can uncheck this box in order to avoid any restriction on who you can share the project with. Now before we can proceed we need to define our storage where all of our files will be stored; this involves two steps. In the first step, we will create a storage service, and then in the second step, we'll have to come back and refresh for the service tip here. Let's do the first step now. So click on Add. All right this page shows the storage service features and the different plans available; we will go with the default free plan, the Lite plan. It should be more than enough for this course. So click create. You can go with the Default settings and just click Confirm. There you go. Now the service has been created, we can go ahead and Refresh for the service to appear. Click Refresh. There you go, and then scroll down and click Create. Awesome, so now the project has been created. So this is the overview page of the project, and now we can start creating notebooks to start building our code. The notebooks are referred to as Assets to this project. You can click on Add to project and select Notebook from the drop-down menu. Now on this page we can decide whether we want to create a Blank notebook, or a notebook from a file, or notebook from URL. Now for this course we're starting from scratch, so we'll have to create a Blank notebook. Let's give it a name. I'm going to name it Capstone_Intro_Notebook and you can give it a brief description. I'm just going to say Capstone Project Demonstration Notebook. We will be using a Python 3 kernel and the default server for runtime. So we scroll down, and we click Create Notebook. It will take some time to initialize the notebook. Right, almost there, and there you go. Now this is a typical Jupyter Notebook, so all the Notebook features can be used here. You can create Markdown cells for example. Let me go ahead and create a title and call this Capstone Project Notebook. Right, you can run code so you can do mathematical operations, so 1+1, or 1+2. It will give you the answer you can execute a print statement, 'I am excited about this course!' One thing that's really cool about these notebooks is that if you have a cell that contains your credentials or other sensitive information and you don't want to share that with either your peers or other collaborators, you can simply add a comment to that cell and it will hide the content. See what you do, add a comment using, @hidden_cell, let's say define my credentials, password one, two, three, four, five, right? When you do that and you share the notebook the content will be hidden, and I'll show you how the cell looks like after you share the notebook. All right, now to share your notebook you have two options. There is direct share using a link to this notebook, and you do that by clicking the share icon. So it opens this mode,l and then you need to activate sharing the notebook and you have three options. You can either share only text and output or you can share all the content excluding sensitive code cells or you share the entire notebook including the code. So just to be safe, it's better to go with the second option, and it tells you what this option means. It means that you can exclude code cells containing sensitive data. The way to share that notebook is you just copy this link, and you paste it somewhere. So you look you will paste it on the Coursera website where you'll be asked to share a link to the notebook or you can email it or send it to any other collaborator you'll be working with. So this is the first option to share a notebook, right? Now the other option which is the one I like to use and recommend is the option to share your notebook by publishing it to your GitHub repository. To do that, you will need to do two things. Number one, create a personal access token and add it to your Watson Studio account, and number two is to link your project to your GitHub repository. Now, before we create the personal access token, let me show you where you have to paste it. So go to your profile icon, and select settings from the drop down menu. Now, on your account settings page, select the Integration's tab, and this is the field where you have to paste your personal access token. Now to create this token, go to your GitHub account, click on your profile image, and select Settings from the dropdown menu, and then on the left menu select Developers Settings and then Personal access tokens. Click Generate new token. Now, give it a name. I'm going to use capstone project token, and I'm going to give myself full access. I'm going to select everything here and then click Generate token. Now, make sure to copy this token because as the message says here you won't be able to see it again. So click on this Copy icon. Now, it's copied. Now, go back to your Watson Studio account and paste the token. Now, click Save and your GitHub account is now linked to your Watson Studio account. So the next thing we have to do is to link the project to a GitHub repository. Now, you can either link it to an existing repository or to start a new and clean repository and use it exclusively for this Capstone course. This is the option that I recommend. You can easily do that by simply going into your GitHub account and by clicking on the Add sign next to your profile image or icon and selecting New repository from the drop down menu. Now, give it a name. I'm going to name it Coursera capstone project, and make sure to check the box to initialize the repository with a README file. It's always a good practice to do that, and then click Create repository. Awesome. So now that the repository is created, let's link our project to it. So let's go back to our Watson Studio account. Click on Projects. Now, you should be able to find your project in the dropdown menu, so click on it. Then go to the settings tab. Now, scroll down to this section that says connect to a GitHub repository, and now enter the address to the repository. So, mine: Cousera capstone project. Once you're done hit Connect, and now your project is linked to the GitHub repository. Now, we should be able to simply push any work that we do, any code, any notebooks that we create to this GitHub repository. So let's go back to the notebook. So we close this, and I'll just go to the Assets tab. You should find your notebook under the Notebooks section. Click the Edit icon. Then to publish this notebook, click on the Publish icon and select Publish on GitHub. Now, make sure that all the settings here are the ones that you're comfortable with, so make sure that all the content is published except for hidden code cells and then click publish, and that's it. Now, you can go to your repository and view the notebook. Let's go to your GitHub account, refresh the page and there you go. Now, note two things. First, the commit comment says that the notebook was published from datascience.ibm.com. This means that you're Watson Studio account was used to publish this notebook. Now, if you try to view the notebook, the hidden cell now shows the code was removed by DSX for sharing. This is super convenient because instead of worrying about deleting cells with sensitive information, you can simply add the hidden cell comment and that will prevent the content from being revealed to your peers or collaborators, and that's it for now. I'll see you in the next video.